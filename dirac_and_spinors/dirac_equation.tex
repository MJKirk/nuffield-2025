\documentclass[a4paper,11pt]{article}

% General LaTeX stuff
\usepackage{geometry} % More reasonable margins
\usepackage[utf8]{inputenc} % utf8 input good for non-ascii characters
\usepackage[T1]{fontenc} % Good for hyphenating non-ascii characters
\usepackage{lmodern}
\usepackage[british]{babel}
\usepackage{graphicx}
\usepackage[pdfusetitle]{hyperref}

% Packages for nicer appearance
\usepackage{microtype} % Improves typesetting
\usepackage[margin=15mm,format=hang]{caption} % Format captions
\usepackage{xfrac} % Nice small fractions
% Ensure all figures are centred by default
\makeatletter
\g@addto@macro\@floatboxreset\centering
\makeatother

% Better maths support
\usepackage[intlimits]{amsmath} % intlimits option places integration limits above and below the integration sign
\usepackage{amssymb} % More maths symbols, e.g. \lesssim
\usepackage{bm} % More bold maths symbols
\usepackage{bbm} % More blackboard bold characters, via \mathbbm. Mostly for blackboard bold 1
\numberwithin{equation}{section} % Reset equation numbers in each section
\allowdisplaybreaks % Allow long maths to be broken over multiple pages

% References and citations
\usepackage{cite} % Combine multiple citations in range
\usepackage[capitalise]{cleveref} % Add automatic reference type text via \cref command
\creflabelformat{equation}{\textup{#2#1#3}} % Removes bracket around equation numbers when referencing

% Better tables
\usepackage{booktabs} % Better tables
\usepackage{array} % Allows us to define custom column types for tables
\newcolumntype{L}{>{$}l<{$}} % A left aligned maths column type

% Physics packages
\usepackage[italic]{hepnames} % Add particle name macros (e.g. \PBs)
% Remove superscript 0 from various meson macros
\renewcommand{\PBs}{\HepParticle{\PB}{\Pqs}{}\xspace}
\renewcommand{\APBs}{\HepAntiParticle{\PB}{\Pqs}{}\xspace}
\renewcommand{\PBd}{\HepParticle{\PB}{\Pqd}{}\xspace}
\usepackage{braket} % Adds Dirac bra-ket notation
\usepackage{slashed} % Adds Dirac slash notation
\usepackage{siunitx} % Add support for units
\sisetup{
	separate-uncertainty, % uncertainties with +- symbol,
	range-phrase = --, % ranges with dash
	range-units = single % only write unit once
}
\DeclareSIUnit\fb{\femto\barn}
\DeclareSIUnit\invps{\ps^{-1}}
\usepackage{suffix} % Allows us to define a starred command, for CKM macro later

% General maths macros
% Redefine Re and Im macros
\renewcommand{\Re}{\operatorname{Re}}
\renewcommand{\Im}{\operatorname{Im}}
\newcommand{\dfk}{\frac{d^4 k}{(2\pi)^4}}
\newcommand{\ddk}{\frac{d^d k}{(2\pi)^d}}
\newcommand{\dfx}{\mathop{d^4 x}}
\newcommand{\BigO}[1]{\ensuremath{\mathcal{O}\left(#1\right)}}

% General HEP macros
\newcommand{\Hamiltonian}{\ensuremath{\mathcal{H}}\xspace}
\newcommand{\Lagrangian}{\ensuremath{\mathcal{L}}\xspace}
\newcommand{\hc}{\text{h.c.}}
\newcommand{\Nc}{\ensuremath{N_c}\xspace}

% Flavour macros
\newcommand{\DeltaM}{\Delta M}
\newcommand{\DeltaG}{\Delta \Gamma}
\newcommand{\asl}{a_\text{sl}}
% Define macros for CKM elements -- the empty superscript on non-starred ensures the subscripts on starred and unstarred are level.
\newcommand{\V}[1]{\ensuremath{V_{#1}^{}}}
\WithSuffix\newcommand\V*[1]{\ensuremath{V_{#1}^*}}

\author{Matthew Kirk}
\title{Solutions of the Dirac equation}

\begin{document}
\maketitle


The Dirac equation is
\begin{equation}
\left( i \gamma^0 \frac{\partial}{\partial t} - i \sum_{j=1}^{3} \left( \gamma^j \frac{\partial}{\partial x_j} \right) - m \mathbbm{1}_{4x4} \right) \psi = 0 \,,
\end{equation}
where $\partial / \partial x_1 = \partial / \partial x, \partial / \partial x_2 = \partial / \partial y, \partial / \partial x_3 = \partial / \partial z$, and $\mathbbm{1}_{4x4}$ is a 4x4 matrix that is all zeroes except for 1's down the diagonal.

In the Dirac representation, the Gamma matrices have the following form:

\begin{align}
\gamma^0 &= \begin{pmatrix}
 1 & 0 & 0 & 0 \\
 0 & 1 & 0 & 0 \\
 0 & 0 & -1 & 0 \\
 0 & 0 & 0 & -1 \\
\end{pmatrix}
\\
\gamma^1 &= \begin{pmatrix}
 0 & 0 & 0 & 0 \\
 0 & 0 & 1 & 0 \\
 0 & -1 & 0 & 0 \\
 -1 & 0 & 0 & 0 \\
\end{pmatrix}
\\
\gamma^2 &= \begin{pmatrix}
 0 & 0 & 0 & -i \\
 0 & 0 & i & 0 \\
 0 & i & 0 & 0 \\
 -i & 0 & 0 & 0 \\
\end{pmatrix}
\\
\gamma^3 &= \begin{pmatrix}
 0 & 0 & 1 & 0 \\
 0 & 0 & 0 & -1 \\
 -1 & 0 & 0 & 0 \\
 0 & 1 & 0 & 0 \\
\end{pmatrix}
\end{align}
\textbf{Exercise:} Show that
\begin{equation}
\psi_1 = \sqrt{E+m} \begin{pmatrix} 1 \\ 0 \\ \frac{p}{E+m} \\ 0 \end{pmatrix} \times e^{-i (E t - p z)}
\end{equation}
is a solution of the Dirac equation, as long as $E^2 = p^2 + m^2$.


In fact, there are four independent solutions of the Dirac equation.
In addition to the one above, there are also the following solutions:
\begin{align}
\psi_2 &= \sqrt{E+m} \begin{pmatrix} 0 \\ 1 \\ 0 \\ \frac{-p}{E+m} \end{pmatrix} \times e^{-i (E t - p z)}
\\
\psi_3 &= \sqrt{E+m} \begin{pmatrix} 0 \\ \frac{-p}{E+m} \\ 0 \\ 1 \end{pmatrix} \times e^{i (E t - p z)}
\\
\psi_4 &= \sqrt{E+m} \begin{pmatrix} \frac{p}{E+m} \\ 0 \\ 1 \\ 0 \end{pmatrix} \times e^{i (E t  -p z)}
\end{align}

% \textbf{Bonus exercise 1:} Show that $\psi_1^\dagger \psi_2 = 0$

\textbf{Bonus exercise:} Show that these other three are also solutions of the Dirac equation.

\end{document}